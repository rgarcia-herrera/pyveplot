\documentclass{bioinfo}
\usepackage{url}
\copyrightyear{2015}
\pubyear{2015}

\begin{document}
\firstpage{1}

\title[short Title]{Pyveplot: SVG Hive Plot API}
\author[]{Rodrigo Garc\'ia-Herrera}
\address{Department of Bioinformatics, National Institute of Genomic
  Medicine, Mexico}

\history{Received on XXXXX; revised on XXXXX; accepted on XXXXX}

\editor{Associate Editor: XXXXXXX}

\maketitle

\begin{abstract}

\section{Sumary:}
Python package provides programmatic object oriented interface for the
creation of Hive plots in Scalable Vector Graphics format.
\section{Availability and Implementation:}
Freely available as a Python package at
https://pypi.python.org/pypi/pyveplot/

\section{Contact:} \href{rgarcia@inmegen.gob.mx}{rgarcia@inmegen.gob.mx}
\end{abstract}

\section{Introduction}

Hive plots are a way of visualizing large networks which allow
for easy assessment and comparison of their properties.
[\cite{krzywinski2012hive}]

There are a few tools available for their creation, some are libraries
for different programming languages, others are Graphical User
Interfaces (GUIs) like ``jhive'' [\cite{jhive}] or ``HiveGraph''
[\cite{hivegraph}]. The latter cover many common use cases such as
plotting directed networks, source-hub-sink categories for nodes,
mapping network structure parameters like degree or centrality to node
size or placement, \&c.

Although GUIs for the creation of Hive plots may be easier to use or
learn, an Application Programming Interface (API) may be better able
to match in descriptive power the ecomplexity of a given plot. This
may be necesary in order to display other aspects of the data, besides
the network structure.

The use of an API may also enable the automation required for batch
creation of plots, for example when the need arises for side-by-side
comparison of networks, a key strength of Hive plots.

Flexibility and automation are boosted by the use of a powerful
programming language such as Python. It is mature and popular, it has
a large number of libraries related to data analysis in general and
life sciences and bioinformatics in particular. Pyveplot is a Python
library that exposes an object oriented API for the creation of Hive
plots. Thus it can readily leverage e.g. the complex network analysis
functionality provided by the powerful NetworkX library
[\cite{hagberg-2008-exploring}], while having no dependencies on it.

In fact, Pyveplot's only dependency is the publicly available
``svgwrite'' library [\cite{svgwrite}]. Scalable Vector Graphics (SVG)
is a modularized format for describing two-dimensional vector
graphics. It is based on XML and developed by the World Wide Web
Consortium [\cite{McCormack:11:SVG}] which means it can be promptly
read by many standards compliant software ranging from editors to
specialized ECMAScript-based language extensions for interactive
visualization, like D3.js [\cite{bostock2011d3}]. While the
requirements of a Hive plot can be met with the most basic shapes
supported by SVG, the design of Pyveplot allows a programmer to take
full advantage of the wide array of capabilities of the graphics
format.

\begin{figure}[th!]
  \centerline{\includegraphics[scale=0.8]{example.pdf}}
  \caption{Hive plot of Erd\"os-R\'enyi graph trivially generated
    using Python's NetworkX. Showing off some SVG goodness: edges and
    axes are path elements that can have any width, color or dash
    pattern; whatever valid SVG element can be used as a node. Any
    number of axes may be defined and placed in the SVG canvas.}
  \label{fig:01}
\end{figure}

The design of the Pyveplot API favors a procedural, imperative way of
creating and linking the elements of a plot, at various degrees of
abstraction. For rich datasets this approach may have better
readability than declarative approaches, which obscure mecanisms in
favor of description of the wanted result, such as are implemented in
libraries like HiveR [\cite{hiveR}].

\section{Object Oriented Approach}



The object oriented API makes the creation of a plot a straightforward
process. A Hive plot consists of:
\begin{itemize}
\item radialy distributed linear axes
\item nodes along those axes
\item conections among those nodes
\end{itemize}
The API provides the corresponding objects: a {\bfseries Hiveplot} object
which contains an arbitrary number of {\bfseries Axis} objects which in
turn contain an arbitrary number of {\bfseries Node} objects, and a method
to connect them.

The \verb"Hiveplot.connect()" method draws edges as B\'ezier curves.
Start and end points map to the coordinates of the given nodes which
in turn are set when adding nodes to an axis with the
\verb"Axis.add_node()" method, by using the placement information of
the axis and a specified offset from its start point.

Control points are set at the same distance from the start (or end)
point of an axis as their corresponding nodes, but along an invisible
axis that shares its origin but diverges by a given angle.

The \verb"Drawing" structural component of the underlying ``svgwrite''
library is available through the \verb"dwg" (drawing) attribute of
{\bfseries Hiveplot}, {\bfseries Axis} and {\bfseries Node} objects.
Any other shape or valid SVG element may be added to them through it.
This allows for a very flexible use of marks, tags, ticks, etc., as is
shown in Figure \ref{fig:01}.

More examples in repository: https://github.com/CSB-IG/pyveplot/


\section*{Acknowledgement}
The author wishes to acknowledge the colaboration of the Computational
and Systems Biology/Integrative Genomics lab at the National Institute
of Genomic Medicine, Mexico.

\bibliographystyle{natbib}
\bibliography{references}

\end{document}
